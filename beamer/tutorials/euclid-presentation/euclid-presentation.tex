\documentclass{beamer}
%\documentclass[handout]{beamer}
\usetheme{Warsaw}
\usecolortheme{seahorse}
\usecolortheme{rose}
\usefonttheme[onlylarge]{structuresmallcapsserif}
\usefonttheme[onlysmall]{structurebold}

\setbeamerfont{title}{shape=\itshape,family=\rmfamily}

\setbeamercolor{title}{fg=red!80!black}
\setbeamercolor{title}{fg=red!80!black,bg=red!20!white}
\mode<presentation>

%\usepackage{pgfpages}
%%\pgfpagesuselayout{resize to}[a4paper,border shrink=5mm,landscape]
%\pgfpagesuselayout{4 on 1}[a4paper,border shrink=5mm,landscape]
\mode<handout>{\setbeamercolor{background canvas}{bg=black!5}}

\title{There Is No Largest Prime Number}
\author[Euclid]{Euclid of Alexandria \\ \texttt{euclid@alexandria.edu}}
\date[ISPN '80]{27th International Symposium of Prime Numbers}

\newtheorem{answeredquestions}[theorem]{Answered Questions}
\newtheorem{openquestions}[theorem]{Open Questions}

\begin{document}

\begin{frame}
	\titlepage
\end{frame}

\begin{frame}
	\frametitle{Outline}
	\tableofcontents
\end{frame}

%\includeonlyframes{frame1}

\section{Motivation}
\subsection{The Basic Problem That We Studies}

\begin{frame}[label=frame1]
	\frametitle{What Are Prime Number?}
	\begin{definition}
		A \alert{prime number} is number that has exactly two divisors.
	\end{definition}
	
	\begin{example}
		\begin{itemize}
			\item 2 is prime (two divisors: 1 and 2).
			\pause
			
			\item 3 is prime (two divisors: 1 and 3).
			\pause
			
			\item 4 is not prime (\alert{three} divisors: 1, 2, and 4).
		\end{itemize}
	\end{example}	
\end{frame}

\begin{frame}[t]
	\frametitle{There Is No Largest Prime Number}
	\framesubtitle{The proof uses \textit{reductio ad absurdum}.}
	
	\begin{proof}
		\begin{enumerate}
			\item<1-> Suppose $p$ were the largest prime number.
			
			\item<2-> Let $q$ the product of the first $p$ number.
			
			\item<3-> Then $q + 1$ not divisiable by any of them.
			
			\item<1-> But $q + 1$ is greater than $1$, thus divisible by some prime number not in the first $p$ numbers. \qedhere			
		\end{enumerate}
	\end{proof}
	
	\uncover<4->{The proof used \textit{reductio ad absurdum}.}
	%\only<4->{The proof used \textit{reductio ad absurdum}.}
\end{frame}

\begin{frame}
	\frametitle{What's Still To Do?}
	
	\begin{block}{Answered Questions}
		How many primes are there?
	\end{block}
	
	\begin{block}{Answered Questions}
		Is every even number the sum of two prime?
	\end{block}
\end{frame}

\begin{frame}
	\frametitle{What’s Still To Do?}
	\begin{itemize}
		\item Answered Questions
		\begin{itemize}
			\item How many primes are there?
		\end{itemize}
		
		\item Open Questions
		\begin{itemize}
			\item Is every even number the sum of two primes?
		\end{itemize}
	\end{itemize}
\end{frame}

\begin{frame}
	\frametitle{What’s Still To Do?}
	\begin{columns}[t]
		\column{.5\textwidth}
		\begin{block}{Answered Questions}
			How many primes are there?
		\end{block}
		
		\column{.5\textwidth}
		\begin{block}{Open Questions}
			Is every even number the sum of two primes?
		\end{block}
	\end{columns}
\end{frame}

\begin{frame}
	\begin{block}{Open Questions}
		Is every even number the sum of two primes?\cite{Goldbach1742}
	\end{block}
\end{frame}

\begin{frame}[fragile]
	\frametitle{An Algorithm For Finding Prime Numbers.}
	\begin{verbatim}
	int main (void)
	{
		std::vector<bool> is_prime (100, true);
		for (int i = 2; i < 100; i++)
			if (is_prime[i])
			{
				std::cout << i << " ";
				for (int j = i; j < 100; is_prime [j] = false, j+=i);
			}
		return 0;
	}
	\end{verbatim}
	\begin{uncoverenv}<2>
		Note the use of \verb|std::|.
	\end{uncoverenv}
\end{frame}

\begin{frame}[fragile]
	\frametitle{An Algorithm For Finding Primes Numbers.}
	\begin{semiverbatim}
		\uncover<1->{\alert<0>{int main (void)}}
		\uncover<1->{\alert<0>{\{}}
		\uncover<1->{\alert<1>{   \alert<4>{std::}vector<bool> is_prime (100, true);}}
		\uncover<1->{\alert<1>{   for (int i = 2; i < 100; i++)}}
		\uncover<2->{\alert<2>{     if (is_prime[i])}}
		\uncover<2->{\alert<0>{      \{}}
		\uncover<3->{\alert<3>{         \alert<4>{std::}cout << i << " ";}}
		\uncover<3->{\alert<3>{          for (int j = i; j < 100;}}
		\uncover<3->{\alert<3>{               is_prime [j] = false, j+=i);}}
		\uncover<2->{\alert<0>{      \}}}
		\uncover<1->{\alert<0>{   return 0;}}
		\uncover<1->{\alert<0>{\}}}
	\end{semiverbatim}
	\visible<4->{Note the use of \alert{\texttt{std::}}.}
\end{frame}

\begin{frame}
	\begin{thebibliography}{10}
		\bibitem{Goldbach1742}[Goldbach, 1742] Christian Goldbach.
		\newblock A problem we should try to solve before the ISPN ’43 deadline,
		\newblock \emph{Letter to Leonhard Euler}, 1742.
	\end{thebibliography}
\end{frame}

\end{document}
